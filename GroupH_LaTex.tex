% Options for packages loaded elsewhere
\PassOptionsToPackage{unicode}{hyperref}
\PassOptionsToPackage{hyphens}{url}
%
\documentclass[
]{article}
\usepackage{amsmath,amssymb}
\usepackage{lmodern}
\usepackage{iftex}
\ifPDFTeX
  \usepackage[T1]{fontenc}
  \usepackage[utf8]{inputenc}
  \usepackage{textcomp} % provide euro and other symbols
\else % if luatex or xetex
  \usepackage{unicode-math}
  \defaultfontfeatures{Scale=MatchLowercase}
  \defaultfontfeatures[\rmfamily]{Ligatures=TeX,Scale=1}
\fi
% Use upquote if available, for straight quotes in verbatim environments
\IfFileExists{upquote.sty}{\usepackage{upquote}}{}
\IfFileExists{microtype.sty}{% use microtype if available
  \usepackage[]{microtype}
  \UseMicrotypeSet[protrusion]{basicmath} % disable protrusion for tt fonts
}{}
\makeatletter
\@ifundefined{KOMAClassName}{% if non-KOMA class
  \IfFileExists{parskip.sty}{%
    \usepackage{parskip}
  }{% else
    \setlength{\parindent}{0pt}
    \setlength{\parskip}{6pt plus 2pt minus 1pt}}
}{% if KOMA class
  \KOMAoptions{parskip=half}}
\makeatother
\usepackage{xcolor}
\usepackage{longtable,booktabs,array}
\usepackage{calc} % for calculating minipage widths
% Correct order of tables after \paragraph or \subparagraph
\usepackage{etoolbox}
\makeatletter
\patchcmd\longtable{\par}{\if@noskipsec\mbox{}\fi\par}{}{}
\makeatother
% Allow footnotes in longtable head/foot
\IfFileExists{footnotehyper.sty}{\usepackage{footnotehyper}}{\usepackage{footnote}}
\makesavenoteenv{longtable}
\usepackage{graphicx}
\makeatletter
\def\maxwidth{\ifdim\Gin@nat@width>\linewidth\linewidth\else\Gin@nat@width\fi}
\def\maxheight{\ifdim\Gin@nat@height>\textheight\textheight\else\Gin@nat@height\fi}
\makeatother
% Scale images if necessary, so that they will not overflow the page
% margins by default, and it is still possible to overwrite the defaults
% using explicit options in \includegraphics[width, height, ...]{}
\setkeys{Gin}{width=\maxwidth,height=\maxheight,keepaspectratio}
% Set default figure placement to htbp
\makeatletter
\def\fps@figure{htbp}
\makeatother
\setlength{\emergencystretch}{3em} % prevent overfull lines
\providecommand{\tightlist}{%
  \setlength{\itemsep}{0pt}\setlength{\parskip}{0pt}}
\setcounter{secnumdepth}{-\maxdimen} % remove section numbering
\ifLuaTeX
  \usepackage{selnolig}  % disable illegal ligatures
\fi
\IfFileExists{bookmark.sty}{\usepackage{bookmark}}{\usepackage{hyperref}}
\IfFileExists{xurl.sty}{\usepackage{xurl}}{} % add URL line breaks if available
\urlstyle{same} % disable monospaced font for URLs
\hypersetup{
  hidelinks,
  pdfcreator={LaTeX via pandoc}}

\author{}
\date{}

\begin{document}

\begin{enumerate}
\def\labelenumi{\Roman{enumi}.}
\item
  \textbf{INTRODUCTION}
\end{enumerate}

Polycystic Ovary Syndrome or PCOS is common in women of childbearing age
with an irregular menstrual cycle and androgen excess. The occurrence of
PCOS in women of reproductive ages today is between five to 20 percent.
It is a serious disease that forms cysts in the ovary due to excess
androgen (male hormone). {[}1{]}

PCOS has been prevalent in endometrial cancer patients who belong to the
premenopausal age group (30-39 years old), and in the Philippines, this
prevalence is apparent currently. In a study of Ortega, G., \& Aguilar,
A., they gathered data from 487 Filipino endometrial cancer patients,
and 61 (12.56\%) have PCOS. 34 out of 61 belong to premenopausal and
menopausal age groups. According to them, Filipino women with
endometrial cancer are more prone to PCOS and are more likely related to
obesity, nulligravida (women who never got pregnant), nulliparous (had
pregnant but never had a child), and records of abnormal blood glucose.
{[}2{]}

Some common symptoms of PCOS include:

\begin{itemize}
\item
  Infrequent or absent menstrual periods
\item
  Excess body hair (hirsutism)
\item
  Acne
\item
  Weight gain or difficulty losing weight
\item
  Balding or thinning hair on the scalp
\item
  Skin tags
\item
  Darkened skin on the neck, groin, or underarms {[}3{]}
\end{itemize}

For this study, an available PCOS dataset from Kaggle repository is used
to train the machine learning models. Since the dataset does classify
the presence of PCOS, the classification accuracy is the important
metric to consider when evaluating the model performance. It is
important to accurately classify whether or not a person has PCOS in
order to provide the appropriate treatment and management. The paper is
arranged as follows: section 2 includes the methodology, sections 3 and
4 are the results and discussion, and conclusion. Section 5 is the
reference.

\begin{enumerate}
\def\labelenumi{\Roman{enumi}.}
\setcounter{enumi}{1}
\item
  \textbf{METHODOLOGY}
\end{enumerate}

In our study of PCOS, we aimed to understand the factors that contribute
to the development of PCOS and to identify potential predictors of the
condition. To accomplish these goals, we used the dataset provided by
Kottarathil. It has 42 attributes of 541 women. We then applied various
statistical techniques, including univariate feature selection and
machine learning algorithms, to identify the most important predictors
of PCOS. We also performed correlation analyses and summary statistics
to better understand the relationships between the different variables
in our dataset. The dataset needs to go through several steps to
preprocess and clean the data.

Figure 1. Block Diagram of the Methodology.

\begin{enumerate}
\def\labelenumi{\arabic{enumi}.}
\item
  \emph{Data Collection}
\end{enumerate}

The first step is to collect the dataset. Multiple online repositories
provide free and available datasets. In this study, the PCOS dataset
provided by Kottarathil from Kaggle will be used. The data was gathered
across 10 different hospitals in Kerala, India. Identities of the female
subjects remained undisclosed.

\begin{enumerate}
\def\labelenumi{\arabic{enumi}.}
\setcounter{enumi}{1}
\item
  \emph{Data Preprocessing}
\end{enumerate}

It is important to preprocess and clean the data to evaluate the
performance without noise. Categorical data were converted to ordinal
values. The dataset was also standardized.

Experimental setups were varied into two setups: (1) setup where all
features are selected, and (2) setup where features selection was
implemented.

\begin{enumerate}
\def\labelenumi{\arabic{enumi}.}
\setcounter{enumi}{2}
\item
  \emph{Experimental Setup 1}

  \begin{enumerate}
  \def\labelenumii{\arabic{enumii}.}
  \item
    \textbf{Fitting into Model and Making Predictions}
  \end{enumerate}
\end{enumerate}

With the data cleaned and selected, it is now ready to be processed by
the models. The models used were KNN, SVM, logistic regression, RF, and
XGboost. Using the prepared models, predictions were made with the
standardized testing set. The k-fold cross-validation approach was used
to determine the performance of each model across different splits of
data.

\begin{enumerate}
\def\labelenumi{\alph{enumi})}
\item
  \textbf{K-Nearest Neighbor}
\end{enumerate}

K-nearest neighbors (KNN) is a simple and effective technique for
classification and regression. It works by storing all available cases
and classifying new cases based on a similarity measure (e.g., distance
functions). Classification is done by a majority vote to its neighbors.
{[}4{]}

\begin{quote}
\includegraphics[width=2.34896in,height=1.67043in]{media/image1.png}Figure
2. Accuracy Vs Neighbor(k) values
\end{quote}

As shown in figure 2, it is concluded that k=10 and k=15 got the highest
accuracy value of \textasciitilde88.23\%, with \textasciitilde96\%
precision and \textasciitilde64\% recall. All features were included.
Therefore, k=10 and k=15 were used on the KNN classifier for this setup.

\includegraphics[width=3.38819in,height=2.409in]{media/image2.png}

Figure 3. Actual values of testing set for KNN

\includegraphics[width=3.38819in,height=2.409in]{media/image3.png}

Figure 4. Predicted values of testing set for KNN

Figure 3 and Figure 4 shows the scatter plot of actual and predicted
values of testing set for KNN with all features. Many a predicted values
differ from the actual values, with an accuracy score of
\textasciitilde88\%, precision \textasciitilde96\%, and
\textasciitilde64\% recall. Which means, KNN only classified
\textasciitilde64\% of the samples correctly.

The mean validation score of KNN is about 86\%. This suggest that the
model is performing consistently in both training and testing set.

\begin{enumerate}
\def\labelenumi{\alph{enumi})}
\setcounter{enumi}{1}
\item
  \textbf{Support Vector Machine}
\end{enumerate}

\includegraphics[width=3.38819in,height=2.40972in]{media/image2.png}SVM
algorithm finds the hyperplane in a high-dimensional space that
maximally separates the different classes.

Figure 5. Actual values of testing set for SVM

\includegraphics[width=3.38819in,height=2.409in]{media/image4.png}

Figure 6. Predicted values of testing set for SVM

Figure 5 and Figure 6 shows the scatter plot of actual and predicted
values of testing set for SVM with all features. Many a predicted values
differ from the actual values, with an accuracy score of
\textasciitilde86\%, precision \textasciitilde87\%, and
\textasciitilde64\% recall. Which means, SVM only classified
\textasciitilde64\% of the samples correctly.

The mean validation score of SVM is about 90\%. This strongly suggest
that the model is performing consistently in both training and testing
set for predicting presence of PCOS.

\begin{enumerate}
\def\labelenumi{\alph{enumi})}
\setcounter{enumi}{2}
\item
  \textbf{Logistic Regression}
\end{enumerate}

Logistic regression is a statistical model that is used to predict a
binary outcome from a set of independent variables. In this study, the
scaled training set was fitted into the model and calculated the
accuracy, precision and recall.

\includegraphics[width=3.38819in,height=2.41011in]{media/image2.png}

\includegraphics[width=3.38819in,height=2.41011in]{media/image5.png}Figure
7. Actual values of testing set for LR

\textbf{Figure 8. Predicted values of testing set for LR}

Figure 7 and Figure 8 shows the scatter plot of actual and predicted
values of testing set for LR with all features. Many a predicted values
differ from the actual values, with an accuracy score of
\textasciitilde84\%, precision \textasciitilde78\%, and
\textasciitilde69\% recall. Which means, LR only classified
\textasciitilde69\% of the samples correctly.

The mean validation score of LR is about 88\%. This suggest that the
model is performing consistently in both training and testing set.

\begin{enumerate}
\def\labelenumi{\alph{enumi})}
\setcounter{enumi}{3}
\item
  \textbf{Random Forest}
\end{enumerate}

The concept behind random forests is that a large number of decision
trees, each trained on a random subset of the training data, can work
together to make more accurate predictions than any individual decision
tree.

Figure 9 and Figure 10 shows the scatter plot of actual and predicted
values of testing set for RF with all features. Many a predicted values
differ from the actual values, with an accuracy score of
\textasciitilde91\%, precision \textasciitilde91\%, and
\textasciitilde78\% recall. Which means, LR only classified
\textasciitilde78\% of the samples correctly. This is the highest
accuracy among the other models for this setup.

The mean validation score of RF is about 91\%. This strongly suggests
that the model is performing consistently in both training and testing
set.

\includegraphics[width=3.38819in,height=2.40972in]{media/image2.png}

\includegraphics[width=3.38819in,height=2.40972in]{media/image6.png}Figure
9. Actual values of testing set for RF

Figure 10. Predicted values of testing set for RF

\begin{enumerate}
\def\labelenumi{\alph{enumi})}
\setcounter{enumi}{4}
\item
  \textbf{Extreme Gradient Boosting}
\end{enumerate}

XGBoost works by building a model in the form of an ensemble of weak
learners (e.g., decision trees), and iteratively improving the model by
adding new weak learners that correct the mistakes of the previous ones.

\includegraphics[width=3.38819in,height=2.40634in]{media/image2.png}

Figure 11. Actual values of testing set for XGB

\includegraphics[width=3.38819in,height=2.40634in]{media/image7.png}

Figure 12. Predicted values of testing set for XGB

Figure 11 and Figure 12 shows the scatter plot of actual and predicted
values of testing set for RF with all features. Few a predicted values
differ from the actual values, with an accuracy score of
\textasciitilde89\%, precision \textasciitilde85\%, and
\textasciitilde80\% recall.

The mean validation score of RF is about 89\%. This suggests that the
model is performing consistently in both training and testing set.

\begin{enumerate}
\def\labelenumi{\arabic{enumi}.}
\setcounter{enumi}{1}
\item
  \textbf{Evaluation of Models}
\end{enumerate}

\includegraphics[width=3.38819in,height=1.84632in]{media/image8.png}

Figure 13. Evaluation metrics for all models

The bar graphs shows a visual comparison of accuracy, precision, and
recall of the different models. RF has the highest accuracy of 0.91 and
XG has the second highest accuracy of 0.89. LR has the lowest accuracy
for this setup with 0.84. In terms of precision, RF also has the highest
score of 0.91. Second highest is KNN with 0.89 and LR having the lowest
precision of 0.78. in terms of recall, XG has the highest with 0.8 and
followed by RF with 0.78. Table 1 shows the numerical comparison of the
evaluation scores of each model.

\begin{longtable}[]{@{}
  >{\raggedright\arraybackslash}p{(\columnwidth - 8\tabcolsep) * \real{0.0760}}
  >{\raggedright\arraybackslash}p{(\columnwidth - 8\tabcolsep) * \real{0.1766}}
  >{\raggedright\arraybackslash}p{(\columnwidth - 8\tabcolsep) * \real{0.2464}}
  >{\raggedright\arraybackslash}p{(\columnwidth - 8\tabcolsep) * \real{0.2464}}
  >{\raggedright\arraybackslash}p{(\columnwidth - 8\tabcolsep) * \real{0.2546}}@{}}
\toprule()
\begin{minipage}[b]{\linewidth}\raggedright
\end{minipage} & \begin{minipage}[b]{\linewidth}\raggedright
\textbf{Model}
\end{minipage} & \begin{minipage}[b]{\linewidth}\raggedright
\textbf{Accuracy}
\end{minipage} & \begin{minipage}[b]{\linewidth}\raggedright
\textbf{Precision}
\end{minipage} & \begin{minipage}[b]{\linewidth}\raggedright
\textbf{Recall}
\end{minipage} \\
\midrule()
\endhead
\textbf{0} & KNN & 0.860294 & 0.896552 & 0.619048 \\
\textbf{1} & SVM & 0.860294 & 0.870968 & 0.642857 \\
\textbf{2} & LR & 0.845588 & 0.783784 & 0.690476 \\
\textbf{3} & RF & 0.911765 & 0.916667 & 0.785714 \\
\textbf{4} & XG & 0.897059 & 0.850000 & 0.809524 \\
\bottomrule()
\end{longtable}

Table 1. Evaluation metrics of models with all features

\begin{enumerate}
\def\labelenumi{\arabic{enumi}.}
\setcounter{enumi}{3}
\item
  \emph{Experimental Setup 2}.

  \begin{enumerate}
  \def\labelenumii{\arabic{enumii}.}
  \item
    \textbf{Feature Selection}
  \end{enumerate}
\end{enumerate}

\begin{quote}
\includegraphics[width=3.38819in,height=2.4125in]{media/image2.png}In
this study, we used the SelectKBest method for feature selection. This
method selects a specified number of the highest scoring features based
on statistical tests. In this case, we selected the top 10 features
based on their p-values, which indicated the probability that the
relationship between the feature and the target variable was due to
chance. There were no significant difference if we selected k number of
features other than 10. This allowed us to narrow down the number of
features in our dataset and improve the interpretability of the models.
The 10 selected features include Weight (Kg), BMI, Cycle(R/I), weight
gain(Y/N), hair growth(Y/N), skin darkening (Y/N), pimples(Y/N), fast
food (Y/N), follicle No. (L), and follicle No. (R), with the top three
are BMI, follicle No. (L), and follicle No. (R). The top three were
identified using the chi-squared approach.
\end{quote}

\begin{enumerate}
\def\labelenumi{\arabic{enumi}.}
\setcounter{enumi}{1}
\item
  \textbf{Fitting into Model and Making Predictions}
\end{enumerate}

With the data cleaned and selected, it is now ready to be processed by
the models. The models used were KNN, SVM, logistic regression, RF, and
XGboost. Using the prepared models, predictions were made with the
standardized testing set. The k-fold cross-validation approach was used
to determine the performance of each model across different splits of
data.

\begin{enumerate}
\def\labelenumi{\alph{enumi})}
\item
  \textbf{K-Nearest Neighbor}
\end{enumerate}

Figure 14. Actual values of testing set for KNN

Figure 14 and Figure 15 shows the scatter plot of actual and predicted
values of testing set for RF with all features. Many a predicted values
differ from the actual values, with an accuracy score of
\textasciitilde88\%, precision \textasciitilde96\%, and
\textasciitilde64\% recall.

The mean validation score of KNN is about 87\%. This suggest that the
model is performing consistently in both training and testing set.

\includegraphics[width=3.38819in,height=2.41271in]{media/image9.png}Figure
15. Predicted values of testing set for KNN

\begin{enumerate}
\def\labelenumi{\alph{enumi})}
\setcounter{enumi}{1}
\item
  \textbf{SVM}
\end{enumerate}

\includegraphics[width=3.38819in,height=2.41271in]{media/image2.png}

Figure 16. Actual values of testing set for SVM

\includegraphics[width=3.38819in,height=2.41271in]{media/image10.png}Figure
17. Predicted values of testing set for SVM

Figure 16 and Figure 17 shows the scatter plot of actual and predicted
values of testing set for SVM with all features. Only a few predicted
values differ from the actual values, with an accuracy score of
\textasciitilde91\%, precision \textasciitilde91\%, and
\textasciitilde80\% recall. Which means, SVM classified
\textasciitilde80\% of the samples correctly.

The mean validation score of SVM is about 90\%. This strongly suggests
that the model is performing consistently in both training and testing
set.

\begin{enumerate}
\def\labelenumi{\alph{enumi})}
\setcounter{enumi}{2}
\item
  \textbf{Logistic Regression}
\end{enumerate}

\includegraphics[width=3.38819in,height=2.41271in]{media/image2.png}Figure
18. Actual values of testing set for LR

\includegraphics[width=3.38819in,height=2.41271in]{media/image11.png}Figure
19. Predicted values of testing set for LR

Figure 18 and Figure 19 shows the scatter plot of actual and predicted
values of testing set for SVM with all features. Only a few predicted
values differ from the actual values, with an accuracy score of
\textasciitilde90\%, precision \textasciitilde87\%, and
\textasciitilde80\% recall. Which means, SVM classified
\textasciitilde80\% of the samples correctly.

The mean validation score of SVM is about 92\%. This strongly suggests
that the model is performing consistently in both training and testing
set.

\begin{enumerate}
\def\labelenumi{\alph{enumi})}
\setcounter{enumi}{3}
\item
  \textbf{Random
  Forest}\includegraphics[width=3.38819in,height=2.4125in]{media/image2.png}
\end{enumerate}

Figure 20. Actual values of testing set for RF

\includegraphics[width=3.38819in,height=2.41271in]{media/image12.png}

Figure 21. Predicted values of testing set for RF

Figure 20 and Figure 21 shows the scatter plot of actual and predicted
values of testing set for SVM with all features. Only a few predicted
values differ from the actual values, with an accuracy score of
\textasciitilde88\%, precision \textasciitilde82\%, and
\textasciitilde80\% recall. Which means, RF classified
\textasciitilde80\% of the samples correctly.

The mean validation score of RF is about 90\%. This suggest that the
model is performing consistently in both training and testing set.

\begin{enumerate}
\def\labelenumi{\alph{enumi})}
\setcounter{enumi}{4}
\item
  \textbf{Extreme Gradient Boosting}
\end{enumerate}

\includegraphics[width=3.38819in,height=2.41271in]{media/image2.png}

Figure 22. Actual values of testing set for RF

\includegraphics[width=3.38819in,height=2.41271in]{media/image12.png}

Figure 23. Predict values of testing set for RF

Figure 22 and Figure 23 shows the scatter plot of actual and predicted
values of testing set for SVM with all features. Many a predicted values
differ from the actual values, with an accuracy score of
\textasciitilde88\%, precision \textasciitilde84\%, and
\textasciitilde78\% recall. Which means, RF classified
\textasciitilde80\% of the samples correctly.

The mean validation score of RF is about 89\%. This suggest that the
model is performing consistently in both training and testing set.

\begin{enumerate}
\def\labelenumi{\arabic{enumi}.}
\item
  \textbf{Evaluation of Models}
\end{enumerate}

\includegraphics[width=3.72203in,height=2.03125in]{media/image13.png}

Figure 24. Evaluation metrics for all models (with feature selection)

\begin{longtable}[]{@{}
  >{\raggedright\arraybackslash}p{(\columnwidth - 8\tabcolsep) * \real{0.0777}}
  >{\raggedright\arraybackslash}p{(\columnwidth - 8\tabcolsep) * \real{0.1807}}
  >{\raggedright\arraybackslash}p{(\columnwidth - 8\tabcolsep) * \real{0.2520}}
  >{\raggedright\arraybackslash}p{(\columnwidth - 8\tabcolsep) * \real{0.2520}}
  >{\raggedright\arraybackslash}p{(\columnwidth - 8\tabcolsep) * \real{0.2375}}@{}}
\toprule()
\begin{minipage}[b]{\linewidth}\raggedright
\end{minipage} & \begin{minipage}[b]{\linewidth}\raggedright
\textbf{Model}
\end{minipage} & \begin{minipage}[b]{\linewidth}\raggedright
\textbf{Accuracy}
\end{minipage} & \begin{minipage}[b]{\linewidth}\raggedright
\textbf{Precision}
\end{minipage} & \begin{minipage}[b]{\linewidth}\raggedright
\textbf{Recall}
\end{minipage} \\
\midrule()
\endhead
\textbf{0} & KNN & 0.852941 & 0.843750 & 0.642857 \\
\textbf{1} & SVM & 0.919118 & 0.918919 & 0.809524 \\
\textbf{2} & LR & 0.904412 & 0.871795 & 0.809524 \\
\textbf{3} & RF & 0.897059 & 0.850000 & 0.809524 \\
\textbf{4} & XG & 0.889706 & 0.846154 & 0.785714 \\
\bottomrule()
\end{longtable}

Table 2. Evaluation metrics for all models (with feature selection)

Figure 24 is a visual comparison of accuracy, precision, and recall of
the different models. SVM has the highest accuracy of 0.91 and LR has
the second highest accuracy of 0.90. KNN has the lowest accuracy for
this setup with 0.85. In terms of precision, SVM also has the highest
score of 0.91. Second highest is LR with 0.87 and KNN having the lowest
precision of 0.84. In terms of recall, SVM, LR, and RF has the highest
with 0.8 and KNN as the lowest with 0.64. Table 1 shows the numerical
comparison of the evaluation scores of each model.

\begin{enumerate}
\def\labelenumi{\arabic{enumi}.}
\setcounter{enumi}{1}
\item
  \textbf{Correlation between selected features}
\end{enumerate}

\includegraphics[width=3.38819in,height=2.64613in]{media/image14.png}

Figure 25. Heatmap of the correlation of different features selected

The feature "BMI'' appears to have a strong positive correlation with
the feature "Weight (Kg)" as indicated by the light shade of orange. On
the other hand, the feature " follicle No. (L)'' appears to have a
strong positive correlation with the feature " follicle No. (R)" as
indicated by the light shade of orange.

\begin{enumerate}
\def\labelenumi{\arabic{enumi}.}
\setcounter{enumi}{4}
\item
  Comparative Analysis of Experimental Setup 1 and Experimental Setup 2
\end{enumerate}

In table 1, random forest achieved the highest accuracy with
\textasciitilde0.91and precision \textasciitilde0.91 among the tested
models in setup 1. Random forest had the second highest recall,
indicating that it was able to correctly identify a significant number
of the positive cases. It is followed by XGBoosting with
\textasciitilde0.89 accuracy, \textasciitilde0.85 precision and
\textasciitilde0.80 recall. Possible implication of this is that all of
the features included in the dataset may be useful for predicting PCOS.
However, in setup 2, the performance of the SVM model improved
significantly, with an accuracy and precision of \textasciitilde0.92 and
\textasciitilde0.81 recall. This could be due to the application of
feature selection, which likely helped the model to focus on the most
relevant features and improve its performance.

We used F1 score to compare the performance of both models. We found
that SVM from setup 2 has better performance than Random Forest from
setup 1 with a F1 score of 0.86 while RF has 0.82. However, XGboosting
has higher F1 score than random forest. It is possible that XGboosting
performs better than random forest.

\begin{enumerate}
\def\labelenumi{\Roman{enumi}.}
\setcounter{enumi}{2}
\item
  \textbf{RESULTS}
\end{enumerate}

The study used 541 samples of data of patients from 10 different
hospitals in India. A total of 42 features were used and out of which
only 10 features were selected using univariate feature selection
(SelectKBest method), with the strong positive correlation between
weight and BMI, and Follicle No. (L) and Follicle No. (R) of female. A
comparison was made between the two different classifiers from different
setups: Random Forest from setup 1, where all features were selected,
and SVM from setup 2 where 10 features were selected. The F1 score
helped determine the best model between the two. The F1 score for RF is
0.825 and for that of SVM is 0.86, hence, model of Support Vector
Machine is selected to determine the absence or presence of PCOS.

\begin{enumerate}
\def\labelenumi{\Roman{enumi}.}
\setcounter{enumi}{3}
\item
  \textbf{CONCLUSION}
\end{enumerate}

In this study, we propose a model of predicting the presence of PCOS in
female. It was shown experimentally that the SVM can be used predicting
the presence of PCOS. In future work, we are very interested in
verifying the robustness of our model by testing it different feature
selection methods. We can also try other machine learning algorithms and
collect more or use different dataset to see if it improves model
performance.

\begin{enumerate}
\def\labelenumi{\Roman{enumi}.}
\setcounter{enumi}{4}
\item
  \textbf{REFERENCES}
\end{enumerate}

\begin{longtable}[]{@{}
  >{\raggedright\arraybackslash}p{(\columnwidth - 0\tabcolsep) * \real{1.0000}}@{}}
\toprule()
\begin{minipage}[b]{\linewidth}\raggedright
{[}1{]}O. Negis, D. Brown, I. Galic, L. Zhaunova, Jain and T. Jain,
"SUN-LB3 Relationship Between BMI and PCOS Symptoms Among Flo App Users
in the United States.," 2020. {[}Online{]}. {[}Accessed 30 December
2022{]}.
\end{minipage} \\
\midrule()
\endhead
{[}2{]}G. Ortega and A. Aguilar, "Prevalence and Characteristics of
Polycystic Ovary Syndrome (PCOS) in Filipino Women Diagnosed with
Endometrial Cancer: A five-year Retrospective Study," \emph{Philippine
Journal of Reproductive Endocrinology and Infertility, ,} p. 13,
2016. \\
{[}3{]}N. I. o. C. H. a. H. Development, "Polycystic Ovary Syndrome
(PCOS)," 2020. {[}Online{]}. Available:
https://www.nichd.nih.gov/health/topics/pcos. {[}Accessed 30 December
2022{]}. \\
{[}4{]}S. Learn, "Nearest Neighbors: Scikit Learn Documentation,"
{[}Online{]}. Available:
https://l.facebook.com/l.php?u=https\%3A\%2F\%2Fscikit-learn.org\%2Fstable\%2Fmodules\%2Fneighbors.html\%3Ffbclid\%3DIwAR3CWJLam9nwAGDsr8rQUClKOMD4bfIp8IvOi2twqXHk-x99lDqBL44Q4L8\&h=AT3jA5o83tiCk4yUKQJzd4t\_ARFQCrNmnwc2LxtD594-wu1iuJkx-day5QB5kP6oo3Wyf7fdD7PoKH3x9U. \\
\bottomrule()
\end{longtable}

\end{document}
